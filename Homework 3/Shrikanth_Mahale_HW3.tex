\documentclass[]{article}
\usepackage{lmodern}
\usepackage{amssymb,amsmath}
\usepackage{ifxetex,ifluatex}
\usepackage{fixltx2e} % provides \textsubscript
\ifnum 0\ifxetex 1\fi\ifluatex 1\fi=0 % if pdftex
  \usepackage[T1]{fontenc}
  \usepackage[utf8]{inputenc}
\else % if luatex or xelatex
  \ifxetex
    \usepackage{mathspec}
  \else
    \usepackage{fontspec}
  \fi
  \defaultfontfeatures{Ligatures=TeX,Scale=MatchLowercase}
\fi
% use upquote if available, for straight quotes in verbatim environments
\IfFileExists{upquote.sty}{\usepackage{upquote}}{}
% use microtype if available
\IfFileExists{microtype.sty}{%
\usepackage[]{microtype}
\UseMicrotypeSet[protrusion]{basicmath} % disable protrusion for tt fonts
}{}
\PassOptionsToPackage{hyphens}{url} % url is loaded by hyperref
\usepackage[unicode=true]{hyperref}
\hypersetup{
            pdftitle={Shrikanth\_Mahale\_HW3\_Part2},
            pdfauthor={Shrikanth Mahale},
            pdfborder={0 0 0},
            breaklinks=true}
\urlstyle{same}  % don't use monospace font for urls
\usepackage[margin=1in]{geometry}
\usepackage{color}
\usepackage{fancyvrb}
\newcommand{\VerbBar}{|}
\newcommand{\VERB}{\Verb[commandchars=\\\{\}]}
\DefineVerbatimEnvironment{Highlighting}{Verbatim}{commandchars=\\\{\}}
% Add ',fontsize=\small' for more characters per line
\usepackage{framed}
\definecolor{shadecolor}{RGB}{248,248,248}
\newenvironment{Shaded}{\begin{snugshade}}{\end{snugshade}}
\newcommand{\KeywordTok}[1]{\textcolor[rgb]{0.13,0.29,0.53}{\textbf{#1}}}
\newcommand{\DataTypeTok}[1]{\textcolor[rgb]{0.13,0.29,0.53}{#1}}
\newcommand{\DecValTok}[1]{\textcolor[rgb]{0.00,0.00,0.81}{#1}}
\newcommand{\BaseNTok}[1]{\textcolor[rgb]{0.00,0.00,0.81}{#1}}
\newcommand{\FloatTok}[1]{\textcolor[rgb]{0.00,0.00,0.81}{#1}}
\newcommand{\ConstantTok}[1]{\textcolor[rgb]{0.00,0.00,0.00}{#1}}
\newcommand{\CharTok}[1]{\textcolor[rgb]{0.31,0.60,0.02}{#1}}
\newcommand{\SpecialCharTok}[1]{\textcolor[rgb]{0.00,0.00,0.00}{#1}}
\newcommand{\StringTok}[1]{\textcolor[rgb]{0.31,0.60,0.02}{#1}}
\newcommand{\VerbatimStringTok}[1]{\textcolor[rgb]{0.31,0.60,0.02}{#1}}
\newcommand{\SpecialStringTok}[1]{\textcolor[rgb]{0.31,0.60,0.02}{#1}}
\newcommand{\ImportTok}[1]{#1}
\newcommand{\CommentTok}[1]{\textcolor[rgb]{0.56,0.35,0.01}{\textit{#1}}}
\newcommand{\DocumentationTok}[1]{\textcolor[rgb]{0.56,0.35,0.01}{\textbf{\textit{#1}}}}
\newcommand{\AnnotationTok}[1]{\textcolor[rgb]{0.56,0.35,0.01}{\textbf{\textit{#1}}}}
\newcommand{\CommentVarTok}[1]{\textcolor[rgb]{0.56,0.35,0.01}{\textbf{\textit{#1}}}}
\newcommand{\OtherTok}[1]{\textcolor[rgb]{0.56,0.35,0.01}{#1}}
\newcommand{\FunctionTok}[1]{\textcolor[rgb]{0.00,0.00,0.00}{#1}}
\newcommand{\VariableTok}[1]{\textcolor[rgb]{0.00,0.00,0.00}{#1}}
\newcommand{\ControlFlowTok}[1]{\textcolor[rgb]{0.13,0.29,0.53}{\textbf{#1}}}
\newcommand{\OperatorTok}[1]{\textcolor[rgb]{0.81,0.36,0.00}{\textbf{#1}}}
\newcommand{\BuiltInTok}[1]{#1}
\newcommand{\ExtensionTok}[1]{#1}
\newcommand{\PreprocessorTok}[1]{\textcolor[rgb]{0.56,0.35,0.01}{\textit{#1}}}
\newcommand{\AttributeTok}[1]{\textcolor[rgb]{0.77,0.63,0.00}{#1}}
\newcommand{\RegionMarkerTok}[1]{#1}
\newcommand{\InformationTok}[1]{\textcolor[rgb]{0.56,0.35,0.01}{\textbf{\textit{#1}}}}
\newcommand{\WarningTok}[1]{\textcolor[rgb]{0.56,0.35,0.01}{\textbf{\textit{#1}}}}
\newcommand{\AlertTok}[1]{\textcolor[rgb]{0.94,0.16,0.16}{#1}}
\newcommand{\ErrorTok}[1]{\textcolor[rgb]{0.64,0.00,0.00}{\textbf{#1}}}
\newcommand{\NormalTok}[1]{#1}
\usepackage{graphicx,grffile}
\makeatletter
\def\maxwidth{\ifdim\Gin@nat@width>\linewidth\linewidth\else\Gin@nat@width\fi}
\def\maxheight{\ifdim\Gin@nat@height>\textheight\textheight\else\Gin@nat@height\fi}
\makeatother
% Scale images if necessary, so that they will not overflow the page
% margins by default, and it is still possible to overwrite the defaults
% using explicit options in \includegraphics[width, height, ...]{}
\setkeys{Gin}{width=\maxwidth,height=\maxheight,keepaspectratio}
\IfFileExists{parskip.sty}{%
\usepackage{parskip}
}{% else
\setlength{\parindent}{0pt}
\setlength{\parskip}{6pt plus 2pt minus 1pt}
}
\setlength{\emergencystretch}{3em}  % prevent overfull lines
\providecommand{\tightlist}{%
  \setlength{\itemsep}{0pt}\setlength{\parskip}{0pt}}
\setcounter{secnumdepth}{0}
% Redefines (sub)paragraphs to behave more like sections
\ifx\paragraph\undefined\else
\let\oldparagraph\paragraph
\renewcommand{\paragraph}[1]{\oldparagraph{#1}\mbox{}}
\fi
\ifx\subparagraph\undefined\else
\let\oldsubparagraph\subparagraph
\renewcommand{\subparagraph}[1]{\oldsubparagraph{#1}\mbox{}}
\fi

% set default figure placement to htbp
\makeatletter
\def\fps@figure{htbp}
\makeatother


\title{Shrikanth\_Mahale\_HW3\_Part2}
\author{Shrikanth Mahale}
\date{4/11/2020}

\begin{document}
\maketitle

Loading Libaries DataExplorer package for exploratory data analysis
Useful Documentation-
\url{https://cran.r-project.org/web/packages/DataExplorer/vignettes/dataexplorer-intro.html}
dplyr - Data Wrangling Package. Check R Learning Guide for resources to
quickly learn dplyr

\begin{Shaded}
\begin{Highlighting}[]
\ControlFlowTok{if}\NormalTok{ (}\OperatorTok{!}\KeywordTok{require}\NormalTok{(dplyr)) }\KeywordTok{install.packages}\NormalTok{(}\StringTok{"dplyr"}\NormalTok{)}
\end{Highlighting}
\end{Shaded}

\begin{verbatim}
## Loading required package: dplyr
\end{verbatim}

\begin{verbatim}
## Warning: package 'dplyr' was built under R version 3.5.3
\end{verbatim}

\begin{verbatim}
## 
## Attaching package: 'dplyr'
\end{verbatim}

\begin{verbatim}
## The following objects are masked from 'package:stats':
## 
##     filter, lag
\end{verbatim}

\begin{verbatim}
## The following objects are masked from 'package:base':
## 
##     intersect, setdiff, setequal, union
\end{verbatim}

\begin{Shaded}
\begin{Highlighting}[]
\KeywordTok{library}\NormalTok{(dplyr)}
\ControlFlowTok{if}\NormalTok{ (}\OperatorTok{!}\KeywordTok{require}\NormalTok{(tidyverse)) }\KeywordTok{install.packages}\NormalTok{(}\StringTok{"tidyverse"}\NormalTok{)}
\end{Highlighting}
\end{Shaded}

\begin{verbatim}
## Loading required package: tidyverse
\end{verbatim}

\begin{verbatim}
## Warning: package 'tidyverse' was built under R version 3.5.3
\end{verbatim}

\begin{verbatim}
## -- Attaching packages ----------------------------------------------------------------------------------------- tidyverse 1.3.0 --
\end{verbatim}

\begin{verbatim}
## v ggplot2 3.3.0     v purrr   0.3.3
## v tibble  3.0.0     v stringr 1.4.0
## v tidyr   1.0.2     v forcats 0.4.0
## v readr   1.3.1
\end{verbatim}

\begin{verbatim}
## Warning: package 'ggplot2' was built under R version 3.5.3
\end{verbatim}

\begin{verbatim}
## Warning: package 'tibble' was built under R version 3.5.3
\end{verbatim}

\begin{verbatim}
## Warning: package 'tidyr' was built under R version 3.5.3
\end{verbatim}

\begin{verbatim}
## Warning: package 'readr' was built under R version 3.5.2
\end{verbatim}

\begin{verbatim}
## Warning: package 'purrr' was built under R version 3.5.3
\end{verbatim}

\begin{verbatim}
## Warning: package 'stringr' was built under R version 3.5.3
\end{verbatim}

\begin{verbatim}
## Warning: package 'forcats' was built under R version 3.5.3
\end{verbatim}

\begin{verbatim}
## -- Conflicts -------------------------------------------------------------------------------------------- tidyverse_conflicts() --
## x dplyr::filter() masks stats::filter()
## x dplyr::lag()    masks stats::lag()
\end{verbatim}

\begin{Shaded}
\begin{Highlighting}[]
\KeywordTok{library}\NormalTok{(tidyverse)}
\ControlFlowTok{if}\NormalTok{ (}\OperatorTok{!}\KeywordTok{require}\NormalTok{(DataExplorer)) }\KeywordTok{install.packages}\NormalTok{(}\StringTok{"DataExplorer"}\NormalTok{)}
\end{Highlighting}
\end{Shaded}

\begin{verbatim}
## Loading required package: DataExplorer
\end{verbatim}

\begin{verbatim}
## Warning: package 'DataExplorer' was built under R version 3.5.3
\end{verbatim}

\begin{Shaded}
\begin{Highlighting}[]
\KeywordTok{library}\NormalTok{(DataExplorer)}
\ControlFlowTok{if}\NormalTok{ (}\OperatorTok{!}\KeywordTok{require}\NormalTok{(ggplot2)) }\KeywordTok{install.packages}\NormalTok{(}\StringTok{"ggplot2"}\NormalTok{)}
\KeywordTok{library}\NormalTok{(ggplot2)}
\end{Highlighting}
\end{Shaded}

Importing Data

\begin{Shaded}
\begin{Highlighting}[]
\NormalTok{data <-}\StringTok{ }\KeywordTok{read.csv}\NormalTok{(}\StringTok{"KAG_data.csv"}\NormalTok{, }\DataTypeTok{stringsAsFactors =} \OtherTok{FALSE}\NormalTok{)}

\NormalTok{data <-}\StringTok{ }\NormalTok{data }\OperatorTok\StringTok{ }\KeywordTok{mutate}\NormalTok{(}\DataTypeTok{CTR =} \KeywordTok{round}\NormalTok{(((Clicks }\OperatorTok{/}\StringTok{ }\NormalTok{Impressions) }\OperatorTok{*}\StringTok{ }\DecValTok{100}\NormalTok{),}\DecValTok{4}\NormalTok{), }
                        \DataTypeTok{CPC =} \KeywordTok{ifelse}\NormalTok{(Clicks }\OperatorTok{!=}\StringTok{ }\DecValTok{0}\NormalTok{, }\KeywordTok{round}\NormalTok{(Spent }\OperatorTok{/}\StringTok{ }\NormalTok{Clicks,}\DecValTok{4}\NormalTok{), Spent), }
                        \DataTypeTok{CostPerConv_Total =} \KeywordTok{ifelse}\NormalTok{(Total_Conversion }\OperatorTok{!=}\DecValTok{0}\NormalTok{,}\KeywordTok{round}\NormalTok{(Spent}\OperatorTok{/}\NormalTok{Total_Conversion,}\DecValTok{4}\NormalTok{),Spent),}
                        \DataTypeTok{CostPerConv_Approved =} \KeywordTok{ifelse}\NormalTok{(Approved_Conversion }\OperatorTok{!=}\DecValTok{0}\NormalTok{,}\KeywordTok{round}\NormalTok{(Spent}\OperatorTok{/}\NormalTok{Approved_Conversion,}\DecValTok{4}\NormalTok{),Spent),}
                        \DataTypeTok{CPM =} \KeywordTok{round}\NormalTok{((Spent }\OperatorTok{/}\StringTok{ }\NormalTok{Impressions) }\OperatorTok{*}\StringTok{ }\DecValTok{1000}\NormalTok{, }\DecValTok{2}\NormalTok{) )}
\end{Highlighting}
\end{Shaded}

Q.1 Which ad (provide ad\_id as the answer) among the ads that have the
least CPC led to the most impressions?

\begin{Shaded}
\begin{Highlighting}[]
\NormalTok{q1 <-}\StringTok{ }\NormalTok{dplyr}\OperatorTok{::}\KeywordTok{arrange}\NormalTok{(data,CPC,}\KeywordTok{desc}\NormalTok{(Impressions))}
\KeywordTok{head}\NormalTok{(q1}\OperatorTok{$}\NormalTok{ad_id,}\DecValTok{1}\NormalTok{)}
\end{Highlighting}
\end{Shaded}

\begin{verbatim}
## [1] 1121094
\end{verbatim}

Q.2 What campaign (provide compaign\_id as the answer) had spent least
efficiently on brand awareness on an average (i.e.~most Cost per mille
or CPM: use total cost for the campaign / total impressions in
thousands)?

\begin{Shaded}
\begin{Highlighting}[]
\NormalTok{q2 <-}\StringTok{ }\NormalTok{dplyr}\OperatorTok{::}\KeywordTok{arrange}\NormalTok{(data,}\KeywordTok{desc}\NormalTok{(CPM))}
\KeywordTok{head}\NormalTok{(q2}\OperatorTok{$}\NormalTok{campaign_id,}\DecValTok{1}\NormalTok{)}
\end{Highlighting}
\end{Shaded}

\begin{verbatim}
## [1] 936
\end{verbatim}

Q.3 Assume each conversion (`Total\_Conversion') is worth \$5, each
approved conversion (`Approved\_Conversion') is worth \$50. ROAS (return
on advertising spent) is revenue as a percentage of the advertising
spent . Calculate ROAS and round it to two decimals. Make a boxplot of
the ROAS grouped by gender for interest = 15, 21, 101 (or interest\_id =
15, 21, 101) in one graph. Also try to use the function `+
scale\_y\_log10()' in ggplot to make the visualization look better (to
do so, you just need to add `+ scale\_y\_log10()' after your ggplot
function). The x-axis label should be `Interest ID' while the y-axis
label should be ROAS. {[}8 points{]}

\begin{Shaded}
\begin{Highlighting}[]
\NormalTok{data <-}\StringTok{ }\NormalTok{data }\OperatorTok\StringTok{ }\KeywordTok{mutate}\NormalTok{(}\DataTypeTok{ROAS =} \KeywordTok{ifelse}\NormalTok{(Spent }\OperatorTok{!=}\FloatTok{0.00}\NormalTok{,}\KeywordTok{round}\NormalTok{((}\FloatTok{5.00}\OperatorTok{*}\NormalTok{Total_Conversion }\OperatorTok{+}\FloatTok{50.00}\OperatorTok{*}\NormalTok{Approved_Conversion)}\OperatorTok{/}\NormalTok{Spent}\OperatorTok{*}\FloatTok{1.00}\NormalTok{,}\DecValTok{2}\NormalTok{),}\FloatTok{0.00}\NormalTok{))}
\NormalTok{data1 <-}\StringTok{ }\NormalTok{data }\OperatorTok
\StringTok{  }\KeywordTok{filter}\NormalTok{(ROAS}\OperatorTok{!=}\OtherTok{Inf}\NormalTok{, interest }\OperatorTok\StringTok{ }\KeywordTok{c}\NormalTok{(}\StringTok{'15'}\NormalTok{,}\StringTok{'21'}\NormalTok{,}\StringTok{'101'}\NormalTok{))}\OperatorTok
\StringTok{  }\KeywordTok{select}\NormalTok{(interest,gender,ROAS)}\CommentTok{#%>%}
\NormalTok{data1}
\end{Highlighting}
\end{Shaded}

\begin{verbatim}
##    interest gender  ROAS
## 1        15      0 41.96
## 2        15      0  1.05
## 3        15      0  2.96
## 4        15      0  0.00
## 5        15      0  1.92
## 6        15      0  0.00
## 7        21      0 35.95
## 8        15      1  4.20
## 9        15      1  3.31
## 10       15      1 53.10
## 11       15      0  1.30
## 12       21      0  0.00
## 13       21      1  0.00
## 14       15      1  0.16
## 15       15      0  0.00
## 16       21      1  0.00
## 17       15      0  0.00
## 18       21      0  0.00
## 19       15      0 47.83
## 20       21      0  0.00
## 21       15      1 23.31
## 22       21      1  0.00
## 23       21      1  0.00
## 24       15      1  0.56
## 25       15      1  0.00
## 26       21      1  0.19
## 27       15      0  0.00
## 28       15      1  2.73
## 29       15      0  5.68
## 30       15      0 18.75
## 31       21      0  0.00
## 32       15      0  0.00
## 33       15      1  0.00
## 34       15      1 31.79
## 35       15      1 11.24
## 36       21      1  4.10
## 37       15      1  3.25
## 38       15      0  3.65
## 39       21      1  0.00
## 40       15      0  0.00
## 41       15      0  0.00
## 42       15      0  0.00
## 43       21      0  5.10
## 44       21      0  0.00
## 45       15      1  0.00
## 46       15      0  2.75
## 47       15      0  2.62
## 48       15      0  1.80
## 49       15      0 37.74
## 50       15      0  3.49
## 51       21      0  7.59
## 52       21      0  0.97
## 53       21      0 28.57
## 54       21      0  7.78
## 55       21      0  5.06
## 56       15      0  1.48
## 57       15      0  1.13
## 58       15      0  0.65
## 59       21      0  0.83
## 60       21      0  1.47
## 61       15      0  1.37
## 62       21      0  0.00
## 63       15      0  5.57
## 64       15      0  0.82
## 65       15      0  0.15
## 66       21      0  0.60
## 67       21      0  0.45
## 68       21      0  0.08
## 69       15      1  1.57
## 70       15      1  1.42
## 71       15      1  0.97
## 72       21      1  4.56
## 73       21      1  2.65
## 74       21      1  2.62
## 75       21      1  3.73
## 76       15      1  1.61
## 77       15      1 20.91
## 78       21      1  1.42
## 79       21      1  0.38
## 80       15      1  0.74
## 81       21      1  3.96
## 82       21      1  0.08
## 83       21      1  0.52
## 84       15      1  0.04
## 85       15      1  0.05
## 86       21      1  0.37
## 87       21      1  0.81
## 88      101      0 29.22
## 89      101      0 18.58
## 90      101      0  0.33
## 91      101      1  2.20
## 92      101      1  0.93
## 93      101      1  2.71
## 94      101      1  1.58
\end{verbatim}

\begin{Shaded}
\begin{Highlighting}[]
\KeywordTok{ggplot}\NormalTok{(}\DataTypeTok{data =}\NormalTok{ data1, }\KeywordTok{aes}\NormalTok{(}\DataTypeTok{x=}\KeywordTok{as.factor}\NormalTok{(interest), }\DataTypeTok{y=}\NormalTok{ROAS)) }\OperatorTok{+}\KeywordTok{geom_boxplot}\NormalTok{(}\KeywordTok{aes}\NormalTok{(}\DataTypeTok{fill=}\KeywordTok{as.factor}\NormalTok{(gender)))}\OperatorTok{+}\StringTok{ }\KeywordTok{scale_y_log10}\NormalTok{()}\OperatorTok{+}\StringTok{ }\KeywordTok{xlab}\NormalTok{(}\StringTok{"Interest"}\NormalTok{) }\OperatorTok{+}\StringTok{ }\KeywordTok{ylab}\NormalTok{(}\StringTok{"ROAS"}\NormalTok{)}\OperatorTok{+}\KeywordTok{guides}\NormalTok{(}\DataTypeTok{fill=}\KeywordTok{guide_legend}\NormalTok{(}\DataTypeTok{title=}\StringTok{"Gender"}\NormalTok{))}
\end{Highlighting}
\end{Shaded}

\begin{verbatim}
## Warning: Transformation introduced infinite values in continuous y-axis
\end{verbatim}

\begin{verbatim}
## Warning: Removed 23 rows containing non-finite values (stat_boxplot).
\end{verbatim}

\includegraphics{Shrikanth_Mahale_HW3_files/figure-latex/unnamed-chunk-5-1.pdf}

\begin{Shaded}
\begin{Highlighting}[]
\NormalTok{data2 <-}\StringTok{ }\NormalTok{data1 }\OperatorTok
\StringTok{  }\KeywordTok{filter}\NormalTok{(interest}\OperatorTok{==}\StringTok{'15'}\NormalTok{, gender}\OperatorTok{==}\StringTok{'0'}\NormalTok{)}
\KeywordTok{boxplot}\NormalTok{(data2}\OperatorTok{$}\NormalTok{ROAS)}
\end{Highlighting}
\end{Shaded}

\includegraphics{Shrikanth_Mahale_HW3_files/figure-latex/unnamed-chunk-5-2.pdf}

\begin{Shaded}
\begin{Highlighting}[]
\NormalTok{data2}
\end{Highlighting}
\end{Shaded}

\begin{verbatim}
##    interest gender  ROAS
## 1        15      0 41.96
## 2        15      0  1.05
## 3        15      0  2.96
## 4        15      0  0.00
## 5        15      0  1.92
## 6        15      0  0.00
## 7        15      0  1.30
## 8        15      0  0.00
## 9        15      0  0.00
## 10       15      0 47.83
## 11       15      0  0.00
## 12       15      0  5.68
## 13       15      0 18.75
## 14       15      0  0.00
## 15       15      0  3.65
## 16       15      0  0.00
## 17       15      0  0.00
## 18       15      0  0.00
## 19       15      0  2.75
## 20       15      0  2.62
## 21       15      0  1.80
## 22       15      0 37.74
## 23       15      0  3.49
## 24       15      0  1.48
## 25       15      0  1.13
## 26       15      0  0.65
## 27       15      0  1.37
## 28       15      0  5.57
## 29       15      0  0.82
## 30       15      0  0.15
\end{verbatim}

\begin{Shaded}
\begin{Highlighting}[]
\NormalTok{dataplot <-}\StringTok{ }\KeywordTok{ggplot}\NormalTok{(}\DataTypeTok{data =}\NormalTok{ data1, }\KeywordTok{aes}\NormalTok{(}\DataTypeTok{x=}\KeywordTok{as.factor}\NormalTok{(interest), }\DataTypeTok{y=}\KeywordTok{as.factor}\NormalTok{(ROAS))) }\OperatorTok{+}\StringTok{ }\KeywordTok{geom_boxplot}\NormalTok{(}\KeywordTok{aes}\NormalTok{(}\DataTypeTok{fill=}\NormalTok{gender)) }
\NormalTok{dataplot <-}\StringTok{ }\NormalTok{dataplot }\OperatorTok{+}\StringTok{ }\KeywordTok{geom_jitter}\NormalTok{()}
\NormalTok{dataplot <-}\StringTok{ }\NormalTok{dataplot }\OperatorTok{+}\StringTok{ }\KeywordTok{facet_wrap}\NormalTok{( }\OperatorTok{~}\StringTok{ }\NormalTok{interest, }\DataTypeTok{scales=}\StringTok{"free"}\NormalTok{)}
\NormalTok{dataplot <-}\StringTok{ }\NormalTok{dataplot }\OperatorTok{+}\StringTok{ }\KeywordTok{xlab}\NormalTok{(}\StringTok{"x-axis"}\NormalTok{) }\OperatorTok{+}\StringTok{ }\KeywordTok{ylab}\NormalTok{(}\StringTok{"y-axis"}\NormalTok{) }\OperatorTok{+}\StringTok{ }\KeywordTok{ggtitle}\NormalTok{(}\StringTok{"Title"}\NormalTok{)}
\NormalTok{dataplot <-}\StringTok{ }\NormalTok{dataplot }\OperatorTok{+}\StringTok{ }\KeywordTok{guides}\NormalTok{(}\DataTypeTok{fill=}\KeywordTok{guide_legend}\NormalTok{(}\DataTypeTok{title=}\StringTok{"Legend_Title"}\NormalTok{))}
\NormalTok{dataplot }
\end{Highlighting}
\end{Shaded}

\includegraphics{Shrikanth_Mahale_HW3_files/figure-latex/unnamed-chunk-5-3.pdf}

Q.4 Summarize the median and mean of ROAS by genders when campign\_id ==
1178.

\begin{Shaded}
\begin{Highlighting}[]
\NormalTok{q4 <-}\StringTok{ }\NormalTok{data }\OperatorTok\StringTok{ }\KeywordTok{filter}\NormalTok{(campaign_id }\OperatorTok{==}\StringTok{ }\DecValTok{1178}\NormalTok{ ) }\OperatorTok\StringTok{ }\KeywordTok{group_by}\NormalTok{(gender) }\OperatorTok\StringTok{ }\KeywordTok{summarise}\NormalTok{(}\DataTypeTok{ROAS_Mean =} \KeywordTok{mean}\NormalTok{(ROAS), }\DataTypeTok{ROAS_Median =} \KeywordTok{median}\NormalTok{(ROAS))}
\NormalTok{q4}
\end{Highlighting}
\end{Shaded}

\begin{verbatim}
## # A tibble: 2 x 3
##   gender ROAS_Mean ROAS_Median
##    <int>     <dbl>       <dbl>
## 1      0      2.49        1.13
## 2      1      1.56        0.7
\end{verbatim}

Loading Libaries

\begin{Shaded}
\begin{Highlighting}[]
\ControlFlowTok{if}\NormalTok{ (}\OperatorTok{!}\KeywordTok{require}\NormalTok{(readr)) }\KeywordTok{install.packages}\NormalTok{(}\StringTok{"readr"}\NormalTok{)}
\KeywordTok{library}\NormalTok{(readr)}
\ControlFlowTok{if}\NormalTok{ (}\OperatorTok{!}\KeywordTok{require}\NormalTok{(correlationfunnel)) }\KeywordTok{install.packages}\NormalTok{(}\StringTok{"correlationfunnel"}\NormalTok{)}
\end{Highlighting}
\end{Shaded}

\begin{verbatim}
## Loading required package: correlationfunnel
\end{verbatim}

\begin{verbatim}
## Warning: package 'correlationfunnel' was built under R version 3.5.3
\end{verbatim}

\begin{verbatim}
## == correlationfunnel Tip #3 ======================================================================================================
## Using `binarize()` with data containing many columns or many rows can increase dimensionality substantially.
## Try subsetting your data column-wise or row-wise to avoid creating too many columns.
## You can always make a big problem smaller by sampling. :)
\end{verbatim}

\begin{Shaded}
\begin{Highlighting}[]
\KeywordTok{library}\NormalTok{(correlationfunnel)}
\ControlFlowTok{if}\NormalTok{ (}\OperatorTok{!}\KeywordTok{require}\NormalTok{(DataExplorer)) }\KeywordTok{install.packages}\NormalTok{(}\StringTok{"DataExplorer"}\NormalTok{)}
\KeywordTok{library}\NormalTok{(DataExplorer)}
\ControlFlowTok{if}\NormalTok{ (}\OperatorTok{!}\KeywordTok{require}\NormalTok{(WVPlots)) }\KeywordTok{install.packages}\NormalTok{(}\StringTok{"WVPlots"}\NormalTok{)}
\end{Highlighting}
\end{Shaded}

\begin{verbatim}
## Loading required package: WVPlots
\end{verbatim}

\begin{verbatim}
## Warning: package 'WVPlots' was built under R version 3.5.3
\end{verbatim}

\begin{Shaded}
\begin{Highlighting}[]
\KeywordTok{library}\NormalTok{(WVPlots)}
\ControlFlowTok{if}\NormalTok{ (}\OperatorTok{!}\KeywordTok{require}\NormalTok{(ggthemes)) }\KeywordTok{install.packages}\NormalTok{(}\StringTok{"ggthemes"}\NormalTok{)}
\end{Highlighting}
\end{Shaded}

\begin{verbatim}
## Loading required package: ggthemes
\end{verbatim}

\begin{verbatim}
## Warning: package 'ggthemes' was built under R version 3.5.3
\end{verbatim}

\begin{Shaded}
\begin{Highlighting}[]
\KeywordTok{library}\NormalTok{(ggthemes) }
\ControlFlowTok{if}\NormalTok{ (}\OperatorTok{!}\KeywordTok{require}\NormalTok{(ROCR)) }\KeywordTok{install.packages}\NormalTok{(}\StringTok{"ROCR"}\NormalTok{)}
\end{Highlighting}
\end{Shaded}

\begin{verbatim}
## Loading required package: ROCR
\end{verbatim}

\begin{verbatim}
## Warning: package 'ROCR' was built under R version 3.5.3
\end{verbatim}

\begin{verbatim}
## Loading required package: gplots
\end{verbatim}

\begin{verbatim}
## Warning: package 'gplots' was built under R version 3.5.3
\end{verbatim}

\begin{verbatim}
## 
## Attaching package: 'gplots'
\end{verbatim}

\begin{verbatim}
## The following object is masked from 'package:stats':
## 
##     lowess
\end{verbatim}

\begin{Shaded}
\begin{Highlighting}[]
\KeywordTok{library}\NormalTok{(ROCR)}
\ControlFlowTok{if}\NormalTok{ (}\OperatorTok{!}\KeywordTok{require}\NormalTok{(caret)) }\KeywordTok{install.packages}\NormalTok{(}\StringTok{"caret"}\NormalTok{)}
\end{Highlighting}
\end{Shaded}

\begin{verbatim}
## Loading required package: caret
\end{verbatim}

\begin{verbatim}
## Warning: package 'caret' was built under R version 3.5.2
\end{verbatim}

\begin{verbatim}
## Loading required package: lattice
\end{verbatim}

\begin{verbatim}
## 
## Attaching package: 'caret'
\end{verbatim}

\begin{verbatim}
## The following object is masked from 'package:purrr':
## 
##     lift
\end{verbatim}

\begin{Shaded}
\begin{Highlighting}[]
\KeywordTok{library}\NormalTok{(caret)}
\ControlFlowTok{if}\NormalTok{ (}\OperatorTok{!}\KeywordTok{require}\NormalTok{(e1071)) }\KeywordTok{install.packages}\NormalTok{(}\StringTok{"e1071"}\NormalTok{)}
\end{Highlighting}
\end{Shaded}

\begin{verbatim}
## Loading required package: e1071
\end{verbatim}

\begin{verbatim}
## Warning: package 'e1071' was built under R version 3.5.3
\end{verbatim}

\begin{Shaded}
\begin{Highlighting}[]
\KeywordTok{library}\NormalTok{(e1071)}
\ControlFlowTok{if}\NormalTok{ (}\OperatorTok{!}\KeywordTok{require}\NormalTok{(corrplot)) }\KeywordTok{install.packages}\NormalTok{(}\StringTok{"corrplot"}\NormalTok{)}
\end{Highlighting}
\end{Shaded}

\begin{verbatim}
## Loading required package: corrplot
\end{verbatim}

\begin{verbatim}
## Warning: package 'corrplot' was built under R version 3.5.3
\end{verbatim}

\begin{verbatim}
## corrplot 0.84 loaded
\end{verbatim}

\begin{Shaded}
\begin{Highlighting}[]
\KeywordTok{library}\NormalTok{(corrplot)}
\end{Highlighting}
\end{Shaded}

Importing Advertising Data

\begin{Shaded}
\begin{Highlighting}[]
\NormalTok{advertising <-}\StringTok{ }\KeywordTok{read.csv}\NormalTok{(}\StringTok{"advertising1.csv"}\NormalTok{, }\DataTypeTok{stringsAsFactors =} \OtherTok{FALSE}\NormalTok{)}
\KeywordTok{head}\NormalTok{(advertising,}\DecValTok{1}\NormalTok{)}
\end{Highlighting}
\end{Shaded}

\begin{verbatim}
##   Daily.Time.Spent.on.Site Age Area.Income Daily.Internet.Usage
## 1                    68.95  35     61833.9               256.09
##                        Ad.Topic.Line        City Male Country
## 1 Cloned 5thgeneration orchestration Wrightburgh    0 Tunisia
##             Timestamp Clicked.on.Ad
## 1 2016-03-27 00:53:11             0
\end{verbatim}

\begin{Shaded}
\begin{Highlighting}[]
\CommentTok{#changing datatype to factor}
\NormalTok{advertising}\OperatorTok{$}\NormalTok{Clicked.on.Ad <-}\StringTok{ }\KeywordTok{as.factor}\NormalTok{(advertising}\OperatorTok{$}\NormalTok{Clicked.on.Ad)}
\KeywordTok{glimpse}\NormalTok{(advertising)}
\end{Highlighting}
\end{Shaded}

\begin{verbatim}
## Rows: 1,000
## Columns: 10
## $ Daily.Time.Spent.on.Site <dbl> 68.95, 80.23, 69.47, 74.15, 68.37, 59...
## $ Age                      <int> 35, 31, 26, 29, 35, 23, 33, 48, 30, 2...
## $ Area.Income              <dbl> 61833.90, 68441.85, 59785.94, 54806.1...
## $ Daily.Internet.Usage     <dbl> 256.09, 193.77, 236.50, 245.89, 225.5...
## $ Ad.Topic.Line            <chr> "Cloned 5thgeneration orchestration",...
## $ City                     <chr> "Wrightburgh", "West Jodi", "Davidton...
## $ Male                     <int> 0, 1, 0, 1, 0, 1, 0, 1, 1, 1, 0, 1, 1...
## $ Country                  <chr> "Tunisia", "Nauru", "San Marino", "It...
## $ Timestamp                <chr> "2016-03-27 00:53:11", "2016-04-04 01...
## $ Clicked.on.Ad            <fct> 0, 0, 0, 0, 0, 0, 0, 1, 0, 0, 1, 0, 1...
\end{verbatim}

Q.5

\begin{enumerate}
\def\labelenumi{\alph{enumi})}
\tightlist
\item
  We aim to explore the dataset so that we can better choose a model to
  implement. Plot histograms for at least 2 of the continuous variables
  in the dataset. Note it is acceptable to plot more than 2. {[}1
  point{]}
\end{enumerate}

\begin{Shaded}
\begin{Highlighting}[]
\KeywordTok{hist}\NormalTok{(advertising}\OperatorTok{$}\NormalTok{Area.Income, }\DataTypeTok{xlab =} \StringTok{"Area Income"}\NormalTok{, }\DataTypeTok{main =} \StringTok{"Histogram of Area Income"}\NormalTok{)}
\end{Highlighting}
\end{Shaded}

\includegraphics{Shrikanth_Mahale_HW3_files/figure-latex/unnamed-chunk-9-1.pdf}

\begin{Shaded}
\begin{Highlighting}[]
\KeywordTok{hist}\NormalTok{(advertising}\OperatorTok{$}\NormalTok{Daily.Internet.Usage, }\DataTypeTok{xlab =} \StringTok{"Daily Internet Usage"}\NormalTok{, }\DataTypeTok{main =} \StringTok{"Histogram of Daily Internet Usage"}\NormalTok{)}
\end{Highlighting}
\end{Shaded}

\includegraphics{Shrikanth_Mahale_HW3_files/figure-latex/unnamed-chunk-9-2.pdf}

\begin{Shaded}
\begin{Highlighting}[]
\KeywordTok{hist}\NormalTok{(advertising}\OperatorTok{$}\NormalTok{Daily.Time.Spent.on.Site, }\DataTypeTok{xlab =} \StringTok{"Daily Time Spent on Site"}\NormalTok{, }\DataTypeTok{main =} \StringTok{"Histogram of Daily Time Spent on Site"}\NormalTok{)}
\end{Highlighting}
\end{Shaded}

\includegraphics{Shrikanth_Mahale_HW3_files/figure-latex/unnamed-chunk-9-3.pdf}

\begin{enumerate}
\def\labelenumi{\alph{enumi})}
\setcounter{enumi}{1}
\tightlist
\item
  Again on the track of exploring the dataset, plot at least 2 bar
  charts reflecting the counts of different values for different
  variables. Note it is acceptable to plot more than 2. {[}1 point{]}
\end{enumerate}

\begin{Shaded}
\begin{Highlighting}[]
\KeywordTok{ggplot}\NormalTok{(advertising, }\KeywordTok{aes}\NormalTok{(}\DataTypeTok{x =}\NormalTok{ Age)) }\OperatorTok{+}\KeywordTok{geom_bar}\NormalTok{()}\OperatorTok{+}\StringTok{ }\KeywordTok{labs}\NormalTok{(}\DataTypeTok{title =} \StringTok{"Bar Plot of Age"}\NormalTok{)}
\end{Highlighting}
\end{Shaded}

\includegraphics{Shrikanth_Mahale_HW3_files/figure-latex/5b-1.pdf}

\begin{Shaded}
\begin{Highlighting}[]
\KeywordTok{ggplot}\NormalTok{(advertising, }\KeywordTok{aes}\NormalTok{(}\DataTypeTok{x =}\NormalTok{ Male)) }\OperatorTok{+}\KeywordTok{geom_bar}\NormalTok{()}\OperatorTok{+}\StringTok{ }\KeywordTok{labs}\NormalTok{(}\DataTypeTok{title =} \StringTok{"Bar Plot of Male"}\NormalTok{)}
\end{Highlighting}
\end{Shaded}

\includegraphics{Shrikanth_Mahale_HW3_files/figure-latex/5b-2.pdf}

\begin{Shaded}
\begin{Highlighting}[]
\KeywordTok{ggplot}\NormalTok{(advertising, }\KeywordTok{aes}\NormalTok{(}\DataTypeTok{x =}\NormalTok{ Clicked.on.Ad)) }\OperatorTok{+}\KeywordTok{geom_bar}\NormalTok{()}\OperatorTok{+}\StringTok{ }\KeywordTok{labs}\NormalTok{(}\DataTypeTok{title =} \StringTok{"Bar Plot of Clicked on Ad"}\NormalTok{)}
\end{Highlighting}
\end{Shaded}

\includegraphics{Shrikanth_Mahale_HW3_files/figure-latex/5b-3.pdf}

\begin{enumerate}
\def\labelenumi{\alph{enumi})}
\setcounter{enumi}{2}
\tightlist
\item
  Plot boxplots for Age, Area.Income, Daily.Internet.Usage and
  Daily.Time.Spent.on.Site separated by the variable Clicked.on.Ad. To
  clarify, we want to create 4 plots, each of which has 2 boxplots: 1
  for people who clicked on the ad, one for those who didn't. {[}2
  points{]}
\end{enumerate}

\begin{Shaded}
\begin{Highlighting}[]
\CommentTok{#Age vs Clicked On Ad}
\KeywordTok{ggplot}\NormalTok{(}\DataTypeTok{data =}\NormalTok{ advertising, }\DataTypeTok{mapping =} \KeywordTok{aes}\NormalTok{(}\DataTypeTok{x =}\NormalTok{ Clicked.on.Ad, }\DataTypeTok{y =}\NormalTok{ Age)) }\OperatorTok{+}\StringTok{ }\KeywordTok{geom_boxplot}\NormalTok{() }\OperatorTok{+}\StringTok{ }\KeywordTok{labs}\NormalTok{(}\DataTypeTok{title =} \StringTok{"Age vs Clicked On Ad"}\NormalTok{, }\DataTypeTok{x =} \StringTok{"Clicked on Ad"}\NormalTok{)}
\end{Highlighting}
\end{Shaded}

\includegraphics{Shrikanth_Mahale_HW3_files/figure-latex/unnamed-chunk-10-1.pdf}

\begin{Shaded}
\begin{Highlighting}[]
\CommentTok{#Area Income vs Clicked On Ad}

\KeywordTok{ggplot}\NormalTok{(}\DataTypeTok{data =}\NormalTok{ advertising, }\DataTypeTok{mapping =} \KeywordTok{aes}\NormalTok{(}\DataTypeTok{x =}\NormalTok{ Clicked.on.Ad, }\DataTypeTok{y =}\NormalTok{ Area.Income)) }\OperatorTok{+}\StringTok{ }\KeywordTok{geom_boxplot}\NormalTok{() }\OperatorTok{+}\StringTok{ }\KeywordTok{labs}\NormalTok{(}\DataTypeTok{title =} \StringTok{"Area Income vs Clicked On Ad"}\NormalTok{, }\DataTypeTok{x =} \StringTok{"Clicked on Ad"}\NormalTok{,}\DataTypeTok{y =} \StringTok{"Area Income"}\NormalTok{)}
\end{Highlighting}
\end{Shaded}

\includegraphics{Shrikanth_Mahale_HW3_files/figure-latex/unnamed-chunk-10-2.pdf}

\begin{Shaded}
\begin{Highlighting}[]
\CommentTok{#Daily Internet Usage vs Clicked On Ad}
\KeywordTok{ggplot}\NormalTok{(}\DataTypeTok{data =}\NormalTok{ advertising, }\DataTypeTok{mapping =} \KeywordTok{aes}\NormalTok{(}\DataTypeTok{x =}\NormalTok{ Clicked.on.Ad, }\DataTypeTok{y =}\NormalTok{ Daily.Internet.Usage)) }\OperatorTok{+}\StringTok{ }\KeywordTok{geom_boxplot}\NormalTok{() }\OperatorTok{+}\StringTok{ }\KeywordTok{labs}\NormalTok{(}\DataTypeTok{title =} \StringTok{"Daily Internet Usage vs Clicked On Ad"}\NormalTok{, }\DataTypeTok{x =} \StringTok{"Clicked on Ad"}\NormalTok{,}\DataTypeTok{y =} \StringTok{"Daily Internet Usage"}\NormalTok{)}
\end{Highlighting}
\end{Shaded}

\includegraphics{Shrikanth_Mahale_HW3_files/figure-latex/unnamed-chunk-10-3.pdf}

\begin{Shaded}
\begin{Highlighting}[]
\CommentTok{#Daily Time Spent on Site vs Clicked On Ad}
\KeywordTok{ggplot}\NormalTok{(}\DataTypeTok{data =}\NormalTok{ advertising, }\DataTypeTok{mapping =} \KeywordTok{aes}\NormalTok{(}\DataTypeTok{x =}\NormalTok{ Clicked.on.Ad, }\DataTypeTok{y =}\NormalTok{ Daily.Time.Spent.on.Site)) }\OperatorTok{+}\StringTok{ }\KeywordTok{geom_boxplot}\NormalTok{() }\OperatorTok{+}\StringTok{ }\KeywordTok{labs}\NormalTok{(}\DataTypeTok{title =} \StringTok{"Daily Time Spent on Site vs Clicked On Ad"}\NormalTok{, }\DataTypeTok{x =} \StringTok{"Clicked on Ad"}\NormalTok{,}\DataTypeTok{y =} \StringTok{"Daily Time Spent on Site"}\NormalTok{)}
\end{Highlighting}
\end{Shaded}

\includegraphics{Shrikanth_Mahale_HW3_files/figure-latex/unnamed-chunk-10-4.pdf}

\begin{enumerate}
\def\labelenumi{\alph{enumi})}
\setcounter{enumi}{3}
\tightlist
\item
  Based on our preliminary boxplots, would you expect an older person to
  be more likely to click on the ad than someone younger? {[}2 points{]}
\end{enumerate}

Answer: Looking at the Age vs Clicked On Ad Box Plot below, the median
age of users clicking the Ad is higher than the Median age of users not
clicking the AD.The maximum age for users clicking the Ad is also higher
than that of the ones not clicking the AD when outliers are ignored.From
this we can conclude that the tendency of older person clicking the AD
is higher.

\begin{Shaded}
\begin{Highlighting}[]
\CommentTok{#Age vs Clicked On Ad}
\KeywordTok{ggplot}\NormalTok{(}\DataTypeTok{data =}\NormalTok{ advertising, }\DataTypeTok{mapping =} \KeywordTok{aes}\NormalTok{(}\DataTypeTok{x =}\NormalTok{ Clicked.on.Ad, }\DataTypeTok{y =}\NormalTok{ Age)) }\OperatorTok{+}\StringTok{ }\KeywordTok{geom_boxplot}\NormalTok{()}\OperatorTok{+}\StringTok{ }\KeywordTok{labs}\NormalTok{(}\DataTypeTok{title =} \StringTok{"Age vs Clicked On Ad"}\NormalTok{, }\DataTypeTok{x =} \StringTok{"Clicked on Ad"}\NormalTok{)}
\end{Highlighting}
\end{Shaded}

\includegraphics{Shrikanth_Mahale_HW3_files/figure-latex/unnamed-chunk-11-1.pdf}

Q.6

Part (a) {[}3 points{]}

\begin{enumerate}
\def\labelenumi{\arabic{enumi}.}
\tightlist
\item
  Make a scatter plot for Area.Income against Age. Separate the
  datapoints by different shapes based on if the datapoint has clicked
  on the ad or not.
\end{enumerate}

\begin{Shaded}
\begin{Highlighting}[]
\KeywordTok{ggplot}\NormalTok{(}\DataTypeTok{data =}\NormalTok{ advertising, }\DataTypeTok{mapping =} \KeywordTok{aes}\NormalTok{(}\DataTypeTok{x =}\NormalTok{ Age, }\DataTypeTok{y =}\NormalTok{ Area.Income)) }\OperatorTok{+}\StringTok{ }\KeywordTok{geom_point}\NormalTok{(}\KeywordTok{aes}\NormalTok{(}\DataTypeTok{shape =}\NormalTok{ Clicked.on.Ad, }\DataTypeTok{color =}\NormalTok{ Clicked.on.Ad))  }\OperatorTok{+}\StringTok{ }\KeywordTok{labs}\NormalTok{(}\DataTypeTok{title =} \StringTok{"Age Vs. Area Income"}\NormalTok{,}\DataTypeTok{y =} \StringTok{"Area Income"}\NormalTok{)}
\end{Highlighting}
\end{Shaded}

\includegraphics{Shrikanth_Mahale_HW3_files/figure-latex/unnamed-chunk-12-1.pdf}

\begin{enumerate}
\def\labelenumi{\arabic{enumi}.}
\setcounter{enumi}{1}
\tightlist
\item
  Based on this plot, would you expect a 31-year-old person with an Area
  income of \$62,000 to click on the ad or not?
\end{enumerate}

Answer: NO . Looking at the scatterplot, we observe that from the Age
Group between 30 to 35 and income above 60,000, the number of clicks ad
has reduced. So I would not expect a 31-year-old person with an Area
income of \$62,000 to click on the ad.

Part (b) {[}3 points{]} 1. Similar to part a), create a scatter plot for
Daily.Time.Spent.on.Site against Age. Separate the datapoints by
different shapes based on if the datapoint has clicked on the ad or not.

\begin{Shaded}
\begin{Highlighting}[]
\KeywordTok{ggplot}\NormalTok{(}\DataTypeTok{data =}\NormalTok{ advertising, }\DataTypeTok{mapping =} \KeywordTok{aes}\NormalTok{(}\DataTypeTok{x =}\NormalTok{ Age, }\DataTypeTok{y =}\NormalTok{ Daily.Time.Spent.on.Site)) }\OperatorTok{+}\StringTok{ }\KeywordTok{geom_point}\NormalTok{(}\KeywordTok{aes}\NormalTok{(}\DataTypeTok{shape =}\NormalTok{ Clicked.on.Ad, }\DataTypeTok{color =}\NormalTok{ Clicked.on.Ad))  }\OperatorTok{+}\StringTok{ }\KeywordTok{labs}\NormalTok{(}\DataTypeTok{title =} \StringTok{"Age Vs. Daily Time Spent on Site"}\NormalTok{ ,}\DataTypeTok{y =} \StringTok{"Daily Time Spent on Site"}\NormalTok{)}
\end{Highlighting}
\end{Shaded}

\includegraphics{Shrikanth_Mahale_HW3_files/figure-latex/unnamed-chunk-13-1.pdf}

\begin{enumerate}
\def\labelenumi{\arabic{enumi}.}
\setcounter{enumi}{1}
\tightlist
\item
  Based on this plot, would you expect a 50-year-old person who spends
  60 minutes daily on the site to click on the ad or not?
\end{enumerate}

Answer: Yes.

Q.7

Part (a) {[}2 points{]}

\begin{enumerate}
\def\labelenumi{\arabic{enumi}.}
\item
  Now that we have done some exploratory data analysis to get a better
  understanding of our raw data, we can begin to move towards designing
  a model to predict advert clicks.
\item
  Generate a correlation funnel (using the correlation funnel package)
  to see which of the variable in the dataset have the most correlation
  with having clicked the advert.
\end{enumerate}

NOTE: Here we are creating the correlation funnel in regards to HAVING
clicked the advert, rather than not. This will lead to a minor
distinction in your code between the 2 cases. However, it will not
affect your results and subsequent variable selection.

\begin{Shaded}
\begin{Highlighting}[]
\NormalTok{ad4=advertising}
\NormalTok{ad4}\OperatorTok{$}\NormalTok{Age =}\KeywordTok{as.factor}\NormalTok{(ad4}\OperatorTok{$}\NormalTok{Age)}
\NormalTok{ad4}\OperatorTok{$}\NormalTok{Male=}\KeywordTok{as.factor}\NormalTok{(ad4}\OperatorTok{$}\NormalTok{Male)}
\NormalTok{ad_binarized_tbl <-}\StringTok{ }\NormalTok{ad4 }\OperatorTok
\StringTok{  }\KeywordTok{binarize}\NormalTok{()}
\end{Highlighting}
\end{Shaded}

\begin{verbatim}
## Warning: All elements of `...` must be named.
## Did you want `data = c(type, role, source)`?
\end{verbatim}

\begin{Shaded}
\begin{Highlighting}[]
\NormalTok{ad_binarized_corr_tbl <-}\StringTok{ }\NormalTok{ad_binarized_tbl }\OperatorTok
\StringTok{  }\KeywordTok{correlate}\NormalTok{(Clicked.on.Ad__}\DecValTok{1}\NormalTok{)}
\end{Highlighting}
\end{Shaded}

\begin{verbatim}
## Warning: The `x` argument of `as_tibble.matrix()` must have column names if `.name_repair` is omitted as of tibble 2.0.0.
## Using compatibility `.name_repair`.
## This warning is displayed once every 8 hours.
## Call `lifecycle::last_warnings()` to see where this warning was generated.
\end{verbatim}

\begin{Shaded}
\begin{Highlighting}[]
\NormalTok{ad_binarized_corr_tbl}
\end{Highlighting}
\end{Shaded}

\begin{verbatim}
## # A tibble: 57 x 3
##    feature                  bin             correlation
##    <fct>                    <chr>                 <dbl>
##  1 Clicked.on.Ad            0                    -1    
##  2 Clicked.on.Ad            1                     1    
##  3 Daily.Internet.Usage     -Inf_138.83           0.577
##  4 Daily.Time.Spent.on.Site -Inf_51.36            0.568
##  5 Daily.Internet.Usage     218.7925_Inf         -0.508
##  6 Daily.Time.Spent.on.Site 78.5475_Inf          -0.480
##  7 Area.Income              -Inf_47031.8025       0.411
##  8 Daily.Internet.Usage     183.13_218.7925      -0.406
##  9 Daily.Time.Spent.on.Site 68.215_78.5475       -0.370
## 10 Daily.Internet.Usage     138.83_183.13         0.337
## # ... with 47 more rows
\end{verbatim}

\begin{Shaded}
\begin{Highlighting}[]
\NormalTok{ad_binarized_corr_tbl }\OperatorTok
\StringTok{  }\KeywordTok{plot_correlation_funnel}\NormalTok{()}
\end{Highlighting}
\end{Shaded}

\includegraphics{Shrikanth_Mahale_HW3_files/figure-latex/unnamed-chunk-14-1.pdf}

Part (b) {[}2 points{]}

\begin{enumerate}
\def\labelenumi{\arabic{enumi}.}
\tightlist
\item
  Based on the generated correlation funnel, choose the 4 most covarying
  variables (with having clicked the advert) and run a logistic
  regression model for Clicked.on.Ad using these 4 variables. The 4 most
  covarying variable are
\end{enumerate}

\begin{enumerate}
\def\labelenumi{\arabic{enumi})}
\tightlist
\item
  Daily.Time.Spent.on.Site
\item
  Age
\item
  Area.Income
\item
  Daily.Internet.Usage
\end{enumerate}

\begin{Shaded}
\begin{Highlighting}[]
\NormalTok{advertising_logistic_regression <-}\StringTok{ }\KeywordTok{glm}\NormalTok{(}\DataTypeTok{data=}\NormalTok{advertising,Clicked.on.Ad }\OperatorTok{~}\StringTok{ }\NormalTok{Daily.Time.Spent.on.Site}\OperatorTok{+}\NormalTok{Age}\OperatorTok{+}\NormalTok{Area.Income}\OperatorTok{+}\NormalTok{Daily.Internet.Usage, }\DataTypeTok{family =} \StringTok{'binomial'}\NormalTok{)}
\end{Highlighting}
\end{Shaded}

\begin{enumerate}
\def\labelenumi{\arabic{enumi}.}
\setcounter{enumi}{1}
\tightlist
\item
  Output the summary of this model.
\end{enumerate}

\begin{Shaded}
\begin{Highlighting}[]
\KeywordTok{summary}\NormalTok{(advertising_logistic_regression)}
\end{Highlighting}
\end{Shaded}

\begin{verbatim}
## 
## Call:
## glm(formula = Clicked.on.Ad ~ Daily.Time.Spent.on.Site + Age + 
##     Area.Income + Daily.Internet.Usage, family = "binomial", 
##     data = advertising)
## 
## Deviance Residuals: 
##     Min       1Q   Median       3Q      Max  
## -2.4578  -0.1341  -0.0333   0.0167   3.1961  
## 
## Coefficients:
##                            Estimate Std. Error z value Pr(>|z|)    
## (Intercept)               2.713e+01  2.714e+00   9.995  < 2e-16 ***
## Daily.Time.Spent.on.Site -1.919e-01  2.066e-02  -9.291  < 2e-16 ***
## Age                       1.709e-01  2.568e-02   6.655 2.83e-11 ***
## Area.Income              -1.354e-04  1.868e-05  -7.247 4.25e-13 ***
## Daily.Internet.Usage     -6.391e-02  6.745e-03  -9.475  < 2e-16 ***
## ---
## Signif. codes:  0 '***' 0.001 '**' 0.01 '*' 0.05 '.' 0.1 ' ' 1
## 
## (Dispersion parameter for binomial family taken to be 1)
## 
##     Null deviance: 1386.3  on 999  degrees of freedom
## Residual deviance:  182.9  on 995  degrees of freedom
## AIC: 192.9
## 
## Number of Fisher Scoring iterations: 8
\end{verbatim}

Q.8 {[}4 points{]}

Now that we have created our logistic regression model using variables
of significance, we must test the model. When testing such models, it is
always recommended to split the data into a training (from which we
build the model) and test (on which we test the model) set. This is done
to avoid bias, as testing the model on the data from which it is
originally built from is unrepresentative of how the model will perform
on new data. That said, for the case of simplicity, test the model on
the full original dataset. Use type =``response'' to ensure we get the
predicted probabilities of clicking the advert Append the predicted
probabilities to a new column in the original dataset or simply to a new
data frame. The choice is up to you, but ensure you know how to
reference this column of probabilities. Using a threshold of 80\% (0.8),
create a new column in the original dataset that represents if the model
predicts a click or not for that person. Note this means probabilities
above 80\% should be treated as a click prediction. Now using the caret
package, create a confusion matrix for the model predictions and actual
clicks. Note you do not need to graph or plot this confusion matrix. How
many false-negative occurrences do you observe? Recall false negative
means the instances where the model predicts the case to be false when
in reality it is true. For this example, this refers to cases where the
ad is clicked but the model predicts that it isn't

\begin{Shaded}
\begin{Highlighting}[]
\NormalTok{advertising}\OperatorTok{$}\NormalTok{predictreg =}\KeywordTok{predict}\NormalTok{(advertising_logistic_regression, advertising, }\DataTypeTok{type=}\StringTok{"response"}\NormalTok{)}
\NormalTok{advertising}\OperatorTok{$}\NormalTok{predictvalue <-}\StringTok{ }\KeywordTok{ifelse}\NormalTok{(advertising}\OperatorTok{$}\NormalTok{predictreg}\OperatorTok{>}\FloatTok{0.8}\NormalTok{, }\DecValTok{1}\NormalTok{,}\DecValTok{0}\NormalTok{)}
\NormalTok{xtab <-}\StringTok{ }\KeywordTok{table}\NormalTok{(advertising}\OperatorTok{$}\NormalTok{Clicked.on.Ad,advertising}\OperatorTok{$}\NormalTok{predictvalue)}
\KeywordTok{confusionMatrix}\NormalTok{(xtab)}
\end{Highlighting}
\end{Shaded}

\begin{verbatim}
## Confusion Matrix and Statistics
## 
##    
##       0   1
##   0 497   3
##   1  36 464
##                                           
##                Accuracy : 0.961           
##                  95% CI : (0.9471, 0.9721)
##     No Information Rate : 0.533           
##     P-Value [Acc > NIR] : < 2.2e-16       
##                                           
##                   Kappa : 0.922           
##  Mcnemar's Test P-Value : 2.99e-07        
##                                           
##             Sensitivity : 0.9325          
##             Specificity : 0.9936          
##          Pos Pred Value : 0.9940          
##          Neg Pred Value : 0.9280          
##              Prevalence : 0.5330          
##          Detection Rate : 0.4970          
##    Detection Prevalence : 0.5000          
##       Balanced Accuracy : 0.9630          
##                                           
##        'Positive' Class : 0               
## 
\end{verbatim}

\end{document}
